%% March 2018
%%%%%%%%%%%%%%%%%%%%%%%%%%%%%%%%%%%%%%%%%%%%%%%%%%%%%%%%%%%%%%%%%%%%%%%%%%%%
% AGUJournalTemplate.tex: this template file is for articles formatted with LaTeX
%
% This file includes commands and instructions
% given in the order necessary to produce a final output that will
% satisfy AGU requirements, including customized APA reference formatting.
%
% You may copy this file and give it your
% article name, and enter your text.
%
%
% Step 1: Set the \documentclass
%
% There are two options for article format:
%
% PLEASE USE THE DRAFT OPTION TO SUBMIT YOUR PAPERS.
% The draft option produces double spaced output.
%

%% To submit your paper:
\documentclass[draft,linenumbers]{agujournal2018}
\usepackage{apacite}
\usepackage{url} %this package should fix any errors with URLs in refs.
%%%%%%%
% As of 2018 we recommend use of the TrackChanges package to mark revisions.
% The trackchanges package adds five new LaTeX commands:
%
%  \note[editor]{The note}
%  \annote[editor]{Text to annotate}{The note}
%  \add[editor]{Text to add}
%  \remove[editor]{Text to remove}
%  \change[editor]{Text to remove}{Text to add}
%
% complete documentation is here: http://trackchanges.sourceforge.net/
%%%%%%%


%% Enter journal name below.
%% Choose from this list of Journals:
%
% JGR: Atmospheres
% JGR: Biogeosciences
% JGR: Earth Surface
% JGR: Oceans
% JGR: Planets
% JGR: Solid Earth
% JGR: Space Physics
% Global Biogeochemical Cycles
% Geophysical Research Letters
% Paleoceanography and Paleoclimatology
% Radio Science
% Reviews of Geophysics
% Tectonics
% Space Weather
% Water Resources Research
% Geochemistry, Geophysics, Geosystems
% Journal of Advances in Modeling Earth Systems (JAMES)
% Earth's Future
% Earth and Space Science
% Geohealth
%
% ie, \journalname{Water Resources Research}

\journalname{JGR: Atmospheres}


% tightlist command for lists without linebreak
\providecommand{\tightlist}{%
  \setlength{\itemsep}{0pt}\setlength{\parskip}{0pt}}



\usepackage{soulutf8}

\begin{document}


%% ------------------------------------------------------------------------ %%
%  Title
%
% (A title should be specific, informative, and brief. Use
% abbreviations only if they are defined in the abstract. Titles that
% start with general keywords then specific terms are optimized in
% searches)
%
%% ------------------------------------------------------------------------ %%

% Example: \title{This is a test title}

\title{labeleR: an R package to optimize the generation of collection
labels and scientific documents}

%% ------------------------------------------------------------------------ %%
%
%  AUTHORS AND AFFILIATIONS
%
%% ------------------------------------------------------------------------ %%

% Authors are individuals who have significantly contributed to the
% research and preparation of the article. Group authors are allowed, if
% each author in the group is separately identified in an appendix.)

% List authors by first name or initial followed by last name and
% separated by commas. Use \affil{} to number affiliations, and
% \thanks{} for author notes.
% Additional author notes should be indicated with \thanks{} (for
% example, for current addresses).

% Example: \authors{A. B. Author\affil{1}\thanks{Current address, Antartica}, B. C. Author\affil{2,3}, and D. E.
% Author\affil{3,4}\thanks{Also funded by Monsanto.}}

\authors{
Julia G. de Aledo
\affil{1, 2, 3}
Jimena Mateo-Martín
\affil{3}
Francisco Rodríguez-Sánchez
\affil{4}
Ignacio Ramos-Gutiérrez
\affil{3, 4}
}


% \affiliation{1}{First Affiliation}
% \affiliation{2}{Second Affiliation}
% \affiliation{3}{Third Affiliation}
% \affiliation{4}{Fourth Affiliation}

\affiliation{1}{Estación Biológica de Doñana}
\affiliation{2}{Universidad Rey Juan Carlos}
\affiliation{3}{Universidad Autónoma de Madrid}
\affiliation{4}{}
\affiliation{}{Universidad de Sevilla}
%(repeat as many times as is necessary)

%% Corresponding Author:
% Corresponding author mailing address and e-mail address:

% (include name and email addresses of the corresponding author.  More
% than one corresponding author is allowed in this LaTeX file and for
% publication; but only one corresponding author is allowed in our
% editorial system.)

% Example: \correspondingauthor{First and Last Name}{email@address.edu}
\correspondingauthor{Ignacio
Ramos-Gutiérrez}{i.ramos.gutierrez@gmail.com}

%% Keypoints, final entry on title page.

%  List up to three key points (at least one is required)
%  Key Points summarize the main points and conclusions of the article
%  Each must be 100 characters or less with no special characters or punctuation

% Example:
% \begin{keypoints}
% \item	List up to three key points (at least one is required)
% \item	Key Points summarize the main points and conclusions of the article
% \item	Each must be 100 characters or less with no special characters or punctuation
% \end{keypoints}

\begin{keypoints}
\item R
\item Rmarkdown
\item herbaria
\end{keypoints}

%% ------------------------------------------------------------------------ %%
%
%  ABSTRACT
%
% A good abstract will begin with a short description of the problem
% being addressed, briefly describe the new data or analyses, then
% briefly states the main conclusion(s) and how they are supported and
% uncertainties.
%% ------------------------------------------------------------------------ %%

%% \begin{abstract} starts the second page

\begin{abstract}
\texttt{labeleR} is an R package designed to automate the creation of
collection labels and documents for scientific events. It simplifies
repetitive and time-consuming tasks, offering a practical alternative to
manual or costly tools. With \texttt{labeleR}, users can generate a wide
variety of customizable PDF documents that can also be automatically
emailed. The package provides a set of functions classified into two
groups: scientific collections (e.g.~labels for herbarium or insects)
and scientific events organization (e.g.~personal badges, abstract books
and certificates of attendance and participation). Starting from a tidy
dataset, users can easily customize content, incorporate QR codes,
logos, images, and edit their own templates. \texttt{labeleR} transforms
tedious and repetitive workflows into an efficient, reproducible
process, contributing to greater scientific productivity. The package is
available under an open-source license and can be freely downloaded from
CRAN or the GitHub repository (https://ecologyr.github.io/labeleR/).
\end{abstract}
\noindent{\bf Plain language summary}\vskip-\parskip

\noindent{Some journals require a plain language summary. See:
https://publications.agu.org/author-resource-center/text-requirements/\#abstract}
\vskip18pt
\section{Cover letter}

\begin{verbatim}
Concise cover letter focused on the question the manuscript attempts to address
\end{verbatim}

\section{Statement of need}

The management and design of scientific labels and event documents is a
time-consuming task. Large-scale label generation tools for herbarium
and scientific collections (used by institutions such as museums or
botanical gardens) are often paid and proprietary software (e.g.
\citet{brahms2025} \citet{irisbg2024}). Microsoft Excel-Word integration
through mailing lists is commonly used at a smaller scale, although
still involving paid software with limited large database management
capacity. Most free alternatives are not open-source, require installing
a program with limited customization, and are often only compatible with
Windows operating system (e.g. \citet{elysia2019} \citet{plabel2020}),
or designed for very specific purposes (e.g. \citet{entomolabels2022}
for insects, \citet{lichenlabler2025} for lichens or \citet{herblabel}
for plant vouchers). Additionally, credentials and certificates for
scientific events are either created manually one at a time, through
paid online servers, or by hiring an event organization company. To our
knowledge, there are no free, customizable tools for the bulk production
and distribution of these documents. \texttt{labeleR} fills this gap
facilitating the creation of scientific collection labels, conference
badges, attendance and participation certificates, and abstract books,
among others.

\section{Package description}

The \texttt{labeleR} package builds upon the \texttt{RMarkdown}
ecosystem (\citet{rmarkdown}) to generate PDF documents from a tidy data
frame in R (Figure 1). \texttt{labeleR} functions include three types of
arguments: (1) R instructions, such as the data object, paths and file
name of the rendered document; (2) ``fixed'' arguments, text that
remains constant across output documents (e.g.~event name or image
path); (3) ``variable'' arguments, linked to columns in the dataframe,
thus changing between documents (e.g.~taxonomic names in labels or
attendee names in certificates). A QR code can be included either
through a fixed argument or a variable argument, without the need for
external software. Users can also edit and adapt the default
\texttt{RMarkdown} templates provided by the package for their own
purposes.

\section{Documents that can be generated with}

\subsection{Labels for collections}

Appropriate labelling of samples is a fundamental step of the scientific
process (i.e., labelling test tubes in laboratories, storing animal or
plant materials or displaying collections in museums or botanical
gardens). A user-friendly bulk rendering tool is vital for efficiently
producing crafted, uniform labels in a reproducible manner. We present
three label types: ``herbarium'' (most complex), ``collection'' (most
aesthetic) and ``tinylabels'' (compact and simplified, for small insect
collections) (Figure 2). These labels can include QR codes (e.g.~links
to websites, images, or identification codes) without additional tools,
making it easy to quickly access and link to external information.

\subsubsection{Herbarium labels}

Herbarium labels are one of the documents with more variable parameters.
Note that the \texttt{family.column} content will always be capitalized,
and the \texttt{taxon.column} one in italics, recommended to be used as
originally defined, while the rest ca be interchangeable. The QR can
stand for a free text (and therefore remain identical in all labels), or
be a column name, and the codes will be rendered with the individual
information of each row. Four different labels will fit in each of the
A4 pdf pages.

\begin{verbatim}
create_herbarium_label(
  data = herbarium.table,
  path = "labeleR_output",
  filename = "herbarium_labels",
  qr = "QR_code",
  title ="Magical flora of the British Isles" ,
  subtitle = "Project: Eliminating plant blindness in Hogwarts students",
  family.column = "Family",
  taxon.column = "Taxon",
  author.column = "Author",
  det.column = "det",
  date.det.column = "Det_date",
  location.column = "Location",
  area.description.column = "Area_description",
  latitude.column = "Latitude",
  longitude.column = "Longitude",
  elevation.column = "Elevation",
  field1.column = "life_form",
  field2.column = "Observations",
  field3.column = "Height",
  collector.column = "Collector",
  collection.column = "Collection_number",
  assistants.column = "Assistants",
  date.column = "Date"
)
\end{verbatim}

\subsubsection{Collection labels}

They count with five variable parameters, which are not recommended to
be too long, along with the possibility of including a QR code (fixed or
variable) and an image (logo or picture). Field 1 will be always
capitalized, and Field 2 italicized. Any field can be left blank. The
user may manually fix the backgroud and text colors to their preference,
using HTML color codes. Eight different labels will fit in each of the
A4 pdf pages.

\begin{verbatim}
create_collection_label(
  data = collection.table,
  path = "labeleR_output",
  filename = "labels",
  qr = "QR_code",
  field1.column = "field1",
  field2.column = "field2",
  field3.column = "field3",
  field4.column = "field6",
  field5.column = "field7",
  system.file("rmarkdown/pictures/Hogwarts_BnW.png", package = "labeleR"),
  bgcolor = "D0ECC1",  #White is "FFFFFF",
  textcolor = "1E3F20" #Black is "000000"
)
\end{verbatim}

\subsubsection{Tiny labels}

This type of labels is a simplified version of the collection label,
including just five variable fields and the possibility of including a
QR code. It is recommended to write short texts in the variable
arguments and in the QR, as they might become difficult to read. 16
different labels will fit in each of the A4 pdf pages.

\begin{verbatim}
create_tiny_label(
  data = tiny.table,
  qr = "QR_code",
  path = "labeleR_output",
  filename = "tinylabels",
  field1.column ="field2",
  field2.column ="field1",
  field3.column ="field3",
  field4.column ="field4",
  field5.column ="field5" 
)
\end{verbatim}

\subsection{Documents for scientific events}

Scientific events often host a high number of participants, and require
the creation of different documentation, such as abstract books,
personal identification badges and certificates for attendees and
participants. Bulk rendering significantly decreases the amount of time
invested in the creation of these documents. Moreover, to deliver
attendance and participation certificates automatically, those
\texttt{labeleR} functions allow users to automatically send individual
documents to email addresses stored in a column.

\subsubsection{Abstract book}

Abstract books result in a single pdf document with multiple pages. Each
abstract will appear on a different page, following the same order as in
the dataframe rows. If other order of appearance is desired, it is
necessary to first arrange the columns in the original dataframe. Each
page will include four variable fields (title, author names,
affiliations and the abstract texts). The output document can include a
table of contents with the titles and page numbers of all abstracts.
Additionally, is possible to insert a custom front page that appearing
at the beginning of the document.

\begin{verbatim}
create_abstractbook(
data=abstract.table,
path = "labeleR_output",
filename = "congress_abstractbook",
title.column = "abstract_title",
authors.column = "authors",
affiliation.column = "affiliation",
text.column = "abstract_text",
title.cex = 20,
authors.cex = 15,
affiliations.cex = 14,
text.cex = 12,
frontpage = "Congress_frontpage.pdf"
)
\end{verbatim}

\subsubsection{Badges}

Badges can be used for personal accreditation in congresses, courses,
meetings, etc. They have only two variable fields (name and
affiliation), and can include two top logos or images. Accreditation
badges include a dot line in the bottom for individual hand-edition once
printed.

\begin{verbatim}
create_badge(
  data = badges.table,
  path = "labeleR_output",
  filename = "badges",
  event = "INTERNATIONAL CONFERENCE OF MUGGLEOLOGY",
  name.column = "List",
  affiliation.column = "Affiliation",
  rpic = system.file("rmarkdown/pictures/Hogwartslogo.png", package = "labeleR"),
  lpic = system.file("rmarkdown/pictures/MinMagic.png", package = "labeleR")
)
\end{verbatim}

\subsubsection{Attendance certificates}

Attendance certificates the only variable parameter is the name of the
attendees. It allows to include a signature as an image, implying that
the signer does not have to sign them individually. This certificate is
available both in English and Spanish.

\begin{verbatim}
create_attendance_certificate(
  data = attendance.table,
  path = "labeleR_output",
  filename = "attendance_certificates",
  language = "English" ,
  name.column = "Names",
  type = "class",
  title = "Potions (year 1992-1993)",
  date = "23/06/1993",
  hours = "200",
  freetext = "taught by Professor S. Snape",
  signer = "A.P.W.B. Dumbledore",
  signer.role = "School Headmaster",
  rpic = system.file("rmarkdown/pictures/Hogwartslogo.png", package = "labeleR"),
  lpic = system.file("rmarkdown/pictures/Hogwartslogo.png", package = "labeleR"),
  signature.pic = system.file("rmarkdown/pictures/dumbledore.png", package = "labeleR")
)
\end{verbatim}

\subsubsection{Participation certificates}

Participation certificates include multiple variable parameters (such as
speaker, affiliation, title, etc.). These documents can be rendered in
English and in Spanish.

\begin{verbatim}
create_participation_certificate(
  data = participation.table,
  path = "labeleR_output",
  filename = "participation_certificates",
  language = "English",
  name.column = "Name",
  affiliation.column = "House",
  comm.type.column = "Comm.type",
  title.column = "Title",
  date.column = "Date",
  type = "online",
  event = "seminar",
  freetext = "organized by Hogwarts School of Magic and Wizardry",
  signer = "A.P.W.B. Dumbledore",
  signer.role = "School Headmaster",
  rpic = system.file("rmarkdown/pictures/Hogwartslogo.png", package = "labeleR"),
  lpic = system.file("rmarkdown/pictures/MinMagic.png", package = "labeleR"),
  signature.pic = system.file("rmarkdown/pictures/dumbledore.png", package = "labeleR")
)
\end{verbatim}

\section{Further applications}

The \texttt{labeleR} philosophy is quite simple: creating multiple
documents with a common design from a dataset containing the required
information. It offers a modular structure that allows for customization
and extension for new applications. For instance, the newly added
\texttt{create\_multichoice} function generates multichoice tests
randomizing the order of questions and possible answers from a given
table (question bank). New developments will happen in the GitHub
repository (https://github.com/EcologyR/labeleR) and eventually pushed
to CRAN. User feedback and code contributions are welcome in the same
repository to keep \texttt{labeleR} as an open and dynamic tool.

\section{Figure legends}

\section{Data Accessibilty Statement}

\section{Competing Interests Statement}

\section{Author Contributions section}

\section{Acknowledgements}

This package has been developed collaboratively between Sevilla and
Madrid, with continuous feedback from colleagues in both locations. We
acknowledge their input, support and collaboration. We are especially
grateful to Manuel Molina for his ideas at `labeleR initial stages. We
would like to emphasize that the original idea has been built
horizontally among early career researchers. Our work has been possible
thanks to the support of the institutions and projects that employ us,
and especially by the software-developing workshop (Cádiz, La Muela,
2023) organized by AEET and funded by US-1381388 grant from Universidad
de Sevilla/Junta de Andalucía/FEDER-UE.

J.G.A. was supported by Next Generation EU Investigo contract
(URJC-AI-17) and ANTENNA Biodiversa+ and European Commission
(PCI2023-146022-2) postdoctoral contract. J.M.M. was supported by the
Comunidad de Madrid and Universidad Autónoma de Madrid doctoral grant
PEJ-2020-AI/AMB-17551 and research assistant contract
PIPF-2022/ECO-24251. F.R.-S. was supported by VI PPIT-US and grants
US-1381388 from Universidad de Sevilla/Junta de Andalucía/FEDER-UE and
CNS2022-135839 funded by MICIU/AEI/10.13039/501100011033 and by European
Union NextGenerationEU/PRTR. I.R.G. was supported by a doctoral grant at
Universidad Autónoma de Madrid and a postdoctoral position at
Universidad de Sevilla (CNS2022-135839).

\bibliography{paper.bib}


\end{document}
